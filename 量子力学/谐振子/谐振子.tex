\documentclass[]{article}
\usepackage{ctex,hyperref}% 输出汉字
\usepackage{amsmath,amssymb,amsfonts}
\usepackage{amsthm,amsmath,amssymb}
\usepackage{mathrsfs}
%opening
\usepackage{setspace}
\usepackage{lipsum}
\usepackage{graphicx}% 图片插入宏包
\usepackage{subfigure}% 并排子图
\usepackage{float}% 浮动环境,用于调整图片位置
\usepackage[export]{adjustbox}% 防止过宽的图片
\usepackage{amsmath}
\usepackage{extarrows}
\graphicspath{{Figures/}}%文章所用图片在当前目录下的 Figures目录
\usepackage{hyperref} %生成引用链接
\usepackage{cleveref} %实现图片和表格、公式的引用
%%链接设置
\hypersetup{colorlinks = false}
\title{谐振子}
\author{步允霆}

\begin{document}
	
	\maketitle
\tableofcontents
\section{谐振子模型}
\subsection{经典力学中的谐振子}
一维谐振子在下述的势场中
\begin{equation}
	V(x)=\dfrac{1}{2}kx^2
\end{equation}
粒子受到的恢复力为:
\begin{equation}
	F_x=-\dfrac{\mathrm{d}V}{\mathrm{d}x}=-kx
\end{equation}
在经典力学中,此粒子的运动在$Ox$轴上的投影是围绕着点$x=0$的正弦型振荡,其角频率为
\begin{equation}
	\omega=\sqrt{\dfrac{k}{m}}
\end{equation}
粒子遵从如下的运动方程
\begin{equation}
	m\dfrac{\mathrm{d}^2x}{\mathrm{d}t^2}=-\dfrac{\mathrm{d}V}{\mathrm{d}x}=-kx
\end{equation}
这个方程的通解具有如下形式
\begin{equation}
	x=x_M\mathrm{cos}(\omega t-\varphi)
	\label{cos}
\end{equation}
粒子的总能量为
\begin{equation}
	E=T+V=\dfrac{p^2}{2m}+\dfrac{1}{2}m\omega^2x^2
\end{equation}
将\eqref{cos}带入上式子得
\begin{equation}
	E=\dfrac{1}{2}m\omega^2x_M^2
\end{equation}
由此可见,粒子的能量与时间无关,且由于$x_M$可以取任意值,故能量可以为任意正数或零。\par 
现在考虑任意的势$V(x)$,他的极小值位于$x=x_0$。将函数$V(x)$在点$x_0$附近展开
\begin{equation}
	V(x)=a+b(x-x_0)^2+c(x-x_0)^3+\cdots
\end{equation}
展开系数为
\begin{align}
	a&=V(x_0)\nonumber\\
	b&=\dfrac{1}{2}\left(\dfrac{\mathrm{d}^2V}{\mathrm{d}x^2} \right) _{x=x_0}\nonumber\\
	c&=\dfrac{1}{3!}\left(\dfrac{\mathrm{d}^3V}{\mathrm{d}x^3} \right) _{x=x_0}
\end{align}

如果粒子在$x_0$附近运动的振幅足够小,以致$(x-x_0)^3$与前面相较可以忽略,那么我们处理的就是一个谐振子问题了,此时动力学方程近似为
\begin{equation}
	F_x=-\dfrac{\mathrm{d}V}{\mathrm{d}x}\simeq-2b(x-x_0)
\end{equation}
其角频率$\omega$与$V(x)$的二阶导数在$x=x_0$处的值有关:
\begin{equation}
	\omega=\sqrt{\dfrac{2b}{m}}=\sqrt{\dfrac{1}{m}\left(\dfrac{\mathrm{d}^2V}{\mathrm{d}x^2} \right) _{x=x_0}}
\end{equation}
由于运动的振幅很小,故谐振子的能量将会很小。
\subsection{哈密顿算符}
在量子力学中,根据量子化规则,用$X$与$P$算符代替$x$与$p$,他们满足关系:
\begin{equation}
	[X,P]=\mathrm{i}\hbar 
	\label{XP-PX}
\end{equation}
于是我们很容易得到体系的哈密顿算符:
\begin{equation}
	H=\dfrac{P^2}{2m}+\dfrac{1}{2}m\omega^2X^2
\end{equation}

由于$H$与时间无关,对谐振子的量子研究归结于求本征方程:
\begin{equation}
	H|\varphi\rangle=E|\varphi\rangle
\end{equation}
在${|x\rangle}$表象中,可以写为:
\begin{equation}
	\left[ -\dfrac{\hbar^2}{2m}\dfrac{\mathrm{d}^2}{\mathrm{d}x^2}+\dfrac{1}{2}m\omega^2x^2\right] \varphi(x)=E\varphi(x)
	\label{eigeneq}
\end{equation}
\section{多项式方法求解本征值方程}
\subsection{二阶线性常微分方程的级数解法}
\subsubsection{二阶线性常微分方程}
形式为:
\begin{equation}
	y^{\prime\prime}+P(x)y^{\prime}+Q(x)y=0
	\label{eqdiff2}
\end{equation}
假设\eqref{eqdiff2}的解可在$x_0$的邻域进行泰勒展开:
\begin{equation}
	y=a_0+a_1(x-x_0)+a_2(x-x_0)^2+a_3(x-x_0)^3+\cdots
\end{equation}
系数为
\begin{align}
	a_0&=y_{(x=x_0)}\nonumber\\
	a_1&=y^{\prime}_{(x=x_0)}\nonumber\\
	a_2&=\dfrac{1}{2}y^{\prime\prime}_{(x=x_0)}\nonumber\\
	a_3&=\dfrac{1}{3!}y^{\prime\prime\prime}_{(x=x_0)}
\end{align}
如果$a_0,a_1$已知,则由\eqref{eqdiff2}
\begin{equation}
	y^{\prime\prime}_{(x=x_0)}=[-P(x)y^{\prime}-Q(x)y]_{(x=x_0)}
\end{equation}
故之后的系数也可以求得。
\subsubsection{级数解法}
如果方程\eqref{eqdiff2}可以写成如下形式:
\begin{equation}
	P(x)y^{\prime\prime}+Q(x)y^{\prime}+R(x)y=0
	\label{eqdiff2rig}
\end{equation}
展开$y$
\begin{equation}
	y=\sum\limits_{k=0}^{\infty}a_kx^k
	\label{taylor1}
\end{equation}
则
\begin{equation}
	y^{\prime}=\sum\limits_{k=0}^{\infty}a_k kx^{k-1}
	\label{taylor2}
\end{equation}
\begin{equation}
	y^{\prime\prime}=\sum\limits_{k=0}^{\infty}a_k k(k-1)x^{k-2}
	\label{taylor3}
\end{equation}
将\eqref{taylor1}\eqref{taylor2}\eqref{taylor3}带入\eqref{eqdiff2rig},则得包含$x$的各幂次的恒等式,每一幂次的系数都必须等于零。由$x^k$的系数为零,可得
\begin{equation}
	a_k=c_1a_{k-1}+c_2a_{k-2}+\cdots+c_{k-1}a_{1}+c_ka_0
	\label{recursion}
\end{equation}
式\eqref{recursion}称为递推公式。\par 
为了比较$x^k$项的系数,我们可以将\eqref{taylor2}\eqref{taylor3}变换成$x^k$项的加和。由\eqref{taylor2}得
\begin{equation}
	y^{\prime}=\sum\limits_{k=0}^{\infty}a_k kx^{k-1}=\sum\limits_{k=1}^{\infty}a_k kx^{k-1}
\end{equation}
令$k^\prime=k-1$,则$k=k^\prime+1$,于是
\begin{equation}
	y^{\prime}=\sum\limits_{k^\prime=0}^{\infty}a_{k^\prime+1} (k^\prime+1)x^{k^\prime}
\end{equation}
将傀标$k^\prime$换为$k$,则
\begin{equation}
	y^{\prime}=\sum\limits_{k=0}^{\infty}a_{k+1} (k+1)x^{k}
\end{equation}
同样,\eqref{taylor3}可以写为
\begin{equation}
	y^{\prime\prime}=\sum\limits_{k=0}^{\infty}a_{k+2} (k+1)(k+2)x^{k}
\end{equation}
\subsubsection{正则奇点邻域的级数解法}
如果微分方程可以写成如下形式
\begin{equation}
	x^2y^{\prime\prime}+xP(x)y^{\prime}+Q(x)y=0
	\label{eqdiff2s}
\end{equation}
当$x=0$,$P(x)$与$Q(x)$是有限的,那么$x=0$就是方程的正则奇点。可以用如下的级数展开
\begin{align}
	y&=x^L\sum\limits_{k=0}^{\infty}a_kx^k\nonumber\\
	y^{\prime}&=x^L\sum\limits_{k=0}^{\infty}a_k (L+k)x^{k-1}\nonumber\\
	y^{\prime\prime}&=x^L\sum\limits_{k=0}^{\infty}a_k (L+k)(L+k-1)x^{k-2}
\end{align}
带入\eqref{eqdiff2s}即可。
\subsection{级数解法求解谐振子}
\subsubsection{函数与变量的变换}
由\eqref{eigeneq},我们引入无量纲算符
\begin{align}
	\widehat{X}&=\beta X\nonumber\\
	\widehat{P}&=\dfrac{P}{\beta\hbar}
\end{align}
参量$\beta$是具有长度的倒数的量纲,定义为
\begin{equation}
	\beta=\sqrt{\dfrac{m\omega}{\hbar}}
\end{equation}
引用$|\xi_{\widehat{x}}\rangle$表示$\widehat{X}$的属于本征值$\widehat{x}$的本征矢:
\begin{equation}
	\widehat{X}|\xi_{\widehat{x}}\rangle=\widehat{x}|\xi_{\widehat{x}}\rangle
\end{equation}
正交归一关系式与封闭性关系式可以写作:
\begin{equation}
	\left\langle \xi_{\widehat{x}}|\xi_{\widehat{x^\prime}}\right\rangle =\delta(\widehat{x}-\widehat{x^\prime})
\end{equation}
\begin{equation}
	\int^{+\infty}_{-\infty}\mathrm{d}\widehat{x}|\xi_{\widehat{x}}\rangle\langle\xi_{\widehat{x}}|=1
	\label{666}
\end{equation}
右矢$|\xi_{\widehat{x}}\rangle$显然是算符$X$的属于本征值$\widehat{x}/\beta $的本征矢,当:
\begin{equation}
	\widehat{x}=\beta x
	\label{777}
\end{equation}
时,右矢$|x\rangle$便与右矢$|\xi_{\widehat{x}}\rangle$成比例。但二者并不相等。实际上,关于右矢$|x\rangle$的封闭性关系式为:
\begin{equation}
	\int^{+\infty}_{-\infty}\mathrm{d}x|x\rangle\langle x|=1
\end{equation}
如果在这个积分中按\eqref{777}进行变量代换,我们便得到
\begin{equation}
	\int^{+\infty}_{-\infty}\dfrac{\mathrm{d}\widehat{x}}{\beta}|\widehat{x}=\beta x\rangle\langle \widehat{x}=\beta x|=1
\end{equation}
与\eqref{666}比较可以看出,例如,我们可以令
\begin{equation}
	|\widehat{x}=\beta x\rangle=\sqrt{\beta}|\xi_{\widehat{x}}\rangle
\end{equation}
便可以使右矢$|\xi_{\widehat{x}}\rangle$作为$\widehat{x}$的函数是正交归一化的,因为右矢$|x\rangle$作为$x$的函数是正交归一化的。\par 
用$|\varphi\rangle$表示任意右矢,则
\begin{equation}
	\widehat{\varphi}(\widehat{x})=\langle\xi_{\widehat{x}}|\varphi\rangle=\dfrac{1}{\sqrt{\beta}}\langle x=\widehat{x}/\beta|\varphi\rangle
\end{equation}
也就是说
\begin{equation}
	\widehat{\varphi}(\widehat{x})=\dfrac{1}{\sqrt{\beta}}\varphi(x=\widehat{x}/\beta)
	\label{1212}
\end{equation}
于是\eqref{eigeneq}可以改写为
\begin{equation}
	\dfrac{1}{2}\left[ \dfrac{\mathrm{d}^2}{\mathrm{d}\widehat{x}^2}+\widehat{x}^2\right] \widehat{\varphi}(\widehat{x})=\varepsilon\widehat{\varphi}(\widehat{x})
	\label{eigeneq15}
\end{equation}
其中
\begin{equation}
	\varepsilon=\dfrac{E}{\hbar\omega}
	\label{1616}
\end{equation}
\subsubsection{多项式方法}
我们可以将\eqref{eigeneq15}写为
\begin{equation}
	\left[ \dfrac{\mathrm{d}^2}{\mathrm{d}\widehat{x}^2}-(\widehat{x}^2-2\varepsilon)\right] \widehat{\varphi}(\widehat{x})=0
	\label{1717}
\end{equation}
考虑$\widehat{x}$非常大时,$\widehat{\varphi}(\widehat{x})$的行为,为此,考虑以下函数
\begin{equation}
	G_\pm(\widehat{x})=e^{\pm\widehat{x}^2/2}
\end{equation}
他是下列微分方程的解
\begin{equation}
	\left[ \dfrac{\mathrm{d}^2}{\mathrm{d}\widehat{x}^2}-(\widehat{x}^2\pm 1)\right] G_\pm(\widehat{x})=0
	\label{1919}
\end{equation}
当$\widehat{x}$趋向无穷大时
\begin{equation}
	\widehat{x}^2\pm 1\sim\widehat{x}^2\sim\widehat{x}^2-2\varepsilon
\end{equation}
于是方程\eqref{1717}和\eqref{1919}渐近于同一形式。从物理上看,有意义的只是处处有界的函数$\widehat{\varphi}(\widehat{x})$,即行为与$e^{-\widehat{x}^2/2}$相同的那些解,于是我们令
\begin{equation}
	\widehat{\varphi}(\widehat{x})=e^{-\widehat{x}^2/2}h(\widehat{x})
	\label{2121}
\end{equation}
将\eqref{2121}带入\eqref{1717},得到
\begin{equation}
	\dfrac{\mathrm{d}^2}{\mathrm{d}\widehat{x}^2}h(\widehat{x})-2\widehat{x}\dfrac{\mathrm{d}}{\mathrm{d}\widehat{x}}h(\widehat{x})+(2\varepsilon-1)h(\widehat{x})=0
	\label{2222}
\end{equation}

$H$的本征函数具有确定的宇称,因为势$V(x)$为一个偶函数,因此\eqref{eigeneq}的解可以在偶函数类或奇函数类中去找。由于$e^{-\widehat{x}^2/2}h(\widehat{x})$是偶函数,因此,我们令
\begin{equation}
	h(\widehat{x})=\sum\limits_{m=0}^{\infty}a_{2m}\widehat{x}^{2m+p}
	\label{ty1}
\end{equation}
其中$a_0$不为零。同时
\begin{equation}
	\dfrac{\mathrm{d}}{\mathrm{d}\widehat{x}}h(\widehat{x})=\sum\limits_{m=0}^{\infty}a_{2m}(2m+p)a_{2m}\widehat{x}^{2m+p-1}
	\label{ty2}
\end{equation}
\begin{equation}
	\dfrac{\mathrm{d}^2}{\mathrm{d}\widehat{x}^2}h(\widehat{x})=\sum\limits_{m=0}^{\infty}a_{2m}(2m+p)(2m+p-1)a_{2m}\widehat{x}^{2m+p-2}
	\label{ty3}
\end{equation}

现在将\eqref{ty1}\eqref{ty2}\eqref{ty3}代入\eqref{2222}。对于一般项$\widehat{x}^{2m+p}$:
\begin{equation}
	(2m+p+2)(2m+p+1)a_{2m+2}=(4m+2p-2\varepsilon+1)a_{2m}
\end{equation}
得到递推公式
\begin{equation}
	a_{2m+2}=\dfrac{4m+2p-2\varepsilon+1}{(2m+p+2)(2m+p+1)}a_{2m}
	\label{2929}
\end{equation}
最低次项是$\widehat{x}^{p-2}$,则
\begin{equation}
	p(p-1)a_0=0
\end{equation}
于是$p=0$[因而$\varphi(x)$是偶函数]或$p=1$[因而$\varphi(x)$是奇函数]。\par 
这样一来,对于任意$\varepsilon$,我们得到子方程\eqref{2222}的两个线性独立的幂级数解,分别对应于$p=0$和$p=1$。\par 
如果级数是无穷级数,从\eqref{2929}可以看出
\begin{equation}
	\dfrac{a_{2m+2}}{a_{2m}}\overset{m\rightarrow\infty}{\sim}\dfrac{1}{m}
\end{equation}
另一方面,函数$e^{\lambda\widehat{x}^2}$的幂级数展开为
\begin{equation}
	e^{\lambda\widehat{x}^2}=\sum\limits_{m=0}^{\infty}b_{2m}\widehat{x}^{2m}
\end{equation}
其中
\begin{equation}
	b_{2m}=\dfrac{\lambda^m}{m!}
\end{equation}
我们有
\begin{equation}
	\dfrac{b_{2m+2}}{b_{2m}}\overset{m\rightarrow\infty}{\sim}\dfrac{\lambda}{m}
\end{equation}
可以看出该级数的渐近行为与$e^{\lambda\widehat{x}^2}$类似,因而当$\widehat{x}\rightarrow\infty$时,$|\widehat{\varphi}(\widehat{x})|$并不是有界的,这样的解没有物理意义,应该舍去。\par 
现在只剩下一种情况,那就是对于$m$的某一数值$m_0$,\eqref{2929}的分子等于零,这样,我们便有:
\begin{equation}
	\begin{cases}
		a_{2m}\neq0& \text{若$m\leqslant m_0$}\\
		a_{2m}=0   & \text{若$m\geqslant m_0$}
	\end{cases}
\end{equation}
此时$h(\widehat{x})$的幂级数退化为$2m_0+p$次的多项式。$\widehat{\varphi}(\widehat{x})$在无穷远处的行为决定于指数函数$e^{-\widehat{x}^2/2}$,于是$\widehat{\varphi}(\widehat{x})$在物理上是合理的(平方可积)。\par 
$m=m_0$时\eqref{2929}的分子为零,这要求下列条件成立
\begin{equation}
	2\varepsilon=2(2m_0+p)+1
	\label{4141}
\end{equation}
如果我们令
\begin{equation}
	2m_0+p=n
\end{equation}
便可以将\eqref{4141}写作
\begin{equation}
	\varepsilon=\varepsilon_n=n+\dfrac{1}{2}
	\label{4343}
\end{equation}
式中$n$为零或任意正整数,这便引入了谐振子能量的量子化,因为\eqref{1616}改写为
\begin{equation}
	E_n=\left(n+\dfrac{1}{2} \right)\hbar\omega 
\end{equation}

同时本征函数可以写为:
\begin{equation}
	\widehat{\varphi}_n(\widehat{x})=e^{-\widehat{x}^2/2}h_n(\widehat{x})
\end{equation}
式中$h_n(\widehat{x})$是一个$n$次多项式。根据\eqref{ty1},$n$为偶数时,$h_n(\widehat{x})$为偶函数;$n$为奇数时,$h_n(\widehat{x})$为奇函数。\par 
基态对应于$n=0$,即$m_0=p=0$,这时$h_n(\widehat{x})$是一个常数,于是
\begin{equation}
	\widehat{\varphi}_n(\widehat{x})=a_0e^{-\widehat{x}^2/2}
\end{equation}
为使$\widehat{\varphi}_n(\widehat{x})$对于$\widehat{x}$归一化,只需取
\begin{equation}
	a_0=\pi^{-1/4}
\end{equation}
再利用\eqref{1212},得
\begin{equation}
	\varphi_0(x)=\left( \dfrac{\beta^2}{\pi}\right)^{1/4}e^{-\beta^2x^2/2} 
\end{equation}

对于第一激发态,$E_1=3/2\hbar\omega$对应于$n=1$,即$m_0=0,p=1$,同上可以看出
\begin{equation}
	\varphi_1(x)=\left( \dfrac{4\beta^6}{\pi}\right)^{1/4}xe^{-\beta^2x^2/2}
\end{equation}

对于任意的$n$,$h_n(\widehat{x})$就是方程\eqref{2222}的多项式解,考虑到量子化条件\eqref{4343},我们可以将方程写作
\begin{equation}
	\left[ \dfrac{\mathrm{d}^2}{\mathrm{d}\widehat{x}^2}-2\widehat{x}\dfrac{\mathrm{d}}{\mathrm{d}\widehat{x}}+2n\right] h_n(\widehat{x})=0
	\label{5252}
\end{equation}
厄米多项式的定义式为
\begin{equation}
	\left[ \dfrac{\mathrm{d}^2}{\mathrm{d}z^2}-2z\dfrac{\mathrm{d}}{\mathrm{d}z}+2n\right] H_n(z)=0
\end{equation}
我们可以看出,\eqref{5252}正是厄米多项式$H_n(\widehat{x})$所满足的微分方程。因此多项式$h_n(\widehat{x})$正比于$H_n(\widehat{x})$,比例因子可以通过$\widehat{\varphi}(\widehat{x})$的归一化来确定。
\section{升降算符法}
本章我们采用另一种方法,升降算符法再次处理谐振子的问题。
\subsection{符号}
\subsubsection{算符$\widehat{X}$和$\widehat{P}$}
观察算符$X,P$是有量纲的,我们可以定义如下没有量纲的算符:
\begin{align}
	\widehat{X}&=\sqrt{\dfrac{m\omega}{\hbar}}X\nonumber\\
	\widehat{P}&=\dfrac{1}{\sqrt{m\hbar\omega}}P
	\label{b1b1}
\end{align}

利用这两个新算符,可以将对易关系式写作
\begin{equation}
	[\widehat{X},\widehat{P}]=\mathrm{i}
\end{equation}
而哈密顿算符为
\begin{equation}
	H=\hbar\omega\widehat{H}
\end{equation}
其中
\begin{equation}
	\widehat{H}=\dfrac{1}{2}(\widehat{X}^2+\widehat{P}^2)
	\label{b4b4}
\end{equation}
于是本征方程为
\begin{equation}
	\widehat{H}|\varphi_\nu^i\rangle=\varepsilon_\nu|\varphi_\nu^i\rangle
	\label{b5b5}
\end{equation}
指标$\nu$即可属于离散集合也可属于连续集合,辅助指标$i$用来区别属于同一本征值$\varepsilon_\nu$的若干互相正交的本征矢。
\subsubsection{算符$a,a^\dagger$以及$N$}
由于$\widehat{X}$和$\widehat{P}$是不可对易的算符,因此$\widehat{X}^2+\widehat{P}^2$并不等于$(\widehat{X}-\mathrm{i}\widehat{P})(\widehat{X}+\mathrm{i}\widehat{P})$。但是,我们只要引入$\widehat{X}-\mathrm{i}\widehat{P}$和与$\widehat{X}+\mathrm{i}\widehat{P}$成正比的算符,就可以使问题大大简化。\par 
因此,令
\begin{subequations}
	\begin{equation}
		a=\dfrac{1}{\sqrt{2}}(\widehat{X}+\mathrm{i}\widehat{P})
	\end{equation}
	\begin{equation}
		a^\dagger=\dfrac{1}{\sqrt{2}}(\widehat{X}-\mathrm{i}\widehat{P})
	\end{equation}
\label{b7b7}
\end{subequations}
反过来:
\begin{subequations}
	\begin{equation}
		\widehat{X}=\dfrac{1}{\sqrt{2}}(a^\dagger+a)
	\end{equation}
	\begin{equation}
		\widehat{P}=\dfrac{\mathrm{i}}{\sqrt{2}}(a^\dagger-a)
	\end{equation}
\end{subequations}
$a$与$a^\dagger$由于因子i,不是厄米算符,他们互为伴随算符。\par 
$a$与$a^\dagger$的对易子为:
\begin{align}
	[a,a^\dagger]&=\dfrac{1}{2}[\widehat{X}+\mathrm{i}\widehat{P},\widehat{X}-\mathrm{i}\widehat{P}]\nonumber\\
				 &=\dfrac{\mathrm{i}}{2}[\widehat{P},\widehat{X}]-\dfrac{\mathrm{i}}{2}[\widehat{X},\widehat{P}]\nonumber\\
				 &=1
				 \label{b9b9}
\end{align}

接下来推导几个重要公式。首先计算$a^\dagger a$
\begin{align}
	a^\dagger a&=\dfrac{1}{2}(\widehat{X}-\mathrm{i}\widehat{P})(\widehat{X}+\mathrm{i}\widehat{P})\nonumber\\
	 &=\dfrac{1}{2}(\widehat{X}^2+\widehat{P}^2+\mathrm{i}\widehat{X}\widehat{P}-\mathrm{i}\widehat{P}\widehat{X})\nonumber\\
	 &=\dfrac{1}{2}(\widehat{X}^2+\widehat{P}^2-1)
\end{align}
与\eqref{b4b4}比较,可以看出
\begin{equation}
	\widehat{H}=a^\dagger a+\dfrac{1}{2}=\dfrac{1}{2}(\widehat{X}-\mathrm{i}\widehat{P})(\widehat{X}+\mathrm{i}\widehat{P})+\dfrac{1}{2}
	\label{b11b11}
\end{equation}
同样可以得出
\begin{equation}
	\widehat{H}=aa^\dagger-+\dfrac{1}{2}
\end{equation}

再引入一个算符$N$,定义为
\begin{equation}
	N=a^\dagger a
	\label{b13b13}
\end{equation}
这是一个厄米算符,此外,根据\eqref{b11b11}
\begin{equation}
	\widehat{H}=N+\dfrac{1}{2}
\end{equation}
由此可见,$\widehat{H}$的本征矢都是$N$的本征矢。\par 
最后再来计算$N$与$a,a^\dagger$的对易子
\begin{subequations}
	\begin{equation}
		[N,a]=[a^\dagger a,a]=a^\dagger[a,a]+[a^\dagger,a]a=-a
		\label{b17a}
	\end{equation}
	\begin{equation}
		[N,a^\dagger]=[a^\dagger a,a^\dagger]=a^\dagger[a,a^\dagger]+[a^\dagger,a^\dagger]a=a^\dagger
		\label{b17b}
	\end{equation}
\end{subequations}

我们可以把\eqref{b5b5}换成$N$的本征值方程
\begin{equation}
	N|\varphi_\nu^i\rangle=\nu|\varphi_\nu^i\rangle
\end{equation}
这个方程一旦解出,我们将会知道$N$的本征矢$|\varphi_\nu^i\rangle$也是$H$的对应于本征值$E_\nu=(\nu+1/2)\hbar\omega|\varphi_\nu^i\rangle$的本征矢:
\begin{equation}
	H|\varphi_\nu^i\rangle=(\nu+1/2)\hbar\omega|\varphi_\nu^i\rangle
	\label{b19b19}
\end{equation}
\subsection{谱的确定}
\subsubsection{引理1-$N$的本征值性质}
算符$N$的本征值$\nu$都是正数或零。\par 
矢量$a|\varphi_\nu^i\rangle$的模的平方为正数或零,即
\begin{equation}
	\|a|\varphi_\nu^i\rangle\|^2=\langle\varphi_\nu^i|a^\dagger a|\varphi_\nu^i\rangle\geqslant0
\end{equation}
根据\eqref{b13b13},便有
\begin{equation}
	\langle\varphi_\nu^i|a^\dagger a|\varphi_\nu^i\rangle=\langle\varphi_\nu^i|N|\varphi_\nu^i\rangle=\nu\langle\varphi_\nu^i|\varphi_\nu^i\rangle
	\label{b21b21}
\end{equation}
由于$\langle\varphi_\nu^i|\varphi_\nu^i\rangle$是正的,便知
\begin{equation}
	\nu\geqslant0
\end{equation}
\subsubsection{引理2-矢量$a|\varphi_\nu^i\rangle$的性质}
假设$|\varphi_\nu^i\rangle$是$N$的非零本征矢,属于本征值$\nu$。\par
——若$\nu=0$,则右矢$a|\varphi_{\nu=0}^i\rangle$为零。\par 
——若$\nu>0$,则右矢$a|\varphi_\nu^i\rangle$是$N$的非零本征矢,属于本征值$\nu-1$。\par 
(1)根据\eqref{b21b21},如果$\nu=0$,那么$a|\varphi_\nu^i\rangle$的模的平方等于零,即该矢量为零。因为,如果$\nu=0$是$N$的本征值,那么,属于这个本征值的任何本征矢$|\varphi_0^i\rangle$都满足下式:
\begin{equation}
	a|\varphi_0^i\rangle=0
\end{equation}

考虑一个矢量$|\varphi\rangle$,假设他满足
\begin{equation}
	a|\varphi\rangle=0
	\label{b23b23}
\end{equation}
用$a^\dagger$左乘此式的两端有:
\begin{equation}
	a^\dagger a|\varphi\rangle=N|\varphi\rangle=0
\end{equation}
因而,满足\eqref{b23b23}的每一个矢量都是$N$的属于本征值$\nu=0$的本征矢。\par 
(2)现在假设$\nu$确定地为正数,根据\eqref{b21b21},既然模的平方不等于零,那么,$a|\varphi_\nu^i\rangle$也不等于零。\par 
我们将算符关系式\eqref{b17a}应用于矢量$|\varphi_\nu^i\rangle$:
\begin{align}
	[N,a]|\varphi_\nu^i\rangle&=-a|\varphi_\nu^i\rangle\nonumber\\
	Na|\varphi_\nu^i\rangle&=aN|\varphi_\nu^i\rangle-a|\varphi_\nu^i\rangle\nonumber\\
	&=a\nu|\varphi_\nu^i\rangle-a|\varphi_\nu^i\rangle\nonumber\\
	N[a|\varphi_\nu^i\rangle]&=(\nu-1)[a|\varphi_\nu^i\rangle]
\end{align}
于是就证明了$a|\varphi_\nu^i\rangle$是$N$的本征矢,属于本征值$\nu-1$。
\subsubsection{引理3-矢量$a^\dagger|\varphi_\nu^i\rangle$的性质}
假设$|\varphi_\nu^i\rangle$是$N$的一个非零本征矢,属于本征值$\nu$。\par 
——$a^\dagger|\varphi_\nu^i\rangle$永远不为零。\par 
——$a^\dagger|\varphi_\nu^i\rangle$是$N$的本征矢,属于本征值$\nu+1$。\par 
(1)利用\eqref{b9b9}\eqref{b13b13}很容易计算矢量$a^\dagger|\varphi_\nu^i\rangle$的模的平方
\begin{align}
	\|a^\dagger|\varphi_\nu^i\rangle\|^2&=\langle\varphi_\nu^i|aa^\dagger|\varphi_\nu^i\rangle\nonumber\\
	&=\langle\varphi_\nu^i|(N+1)|\varphi_\nu^i\rangle\nonumber\\
	&=(\nu+1)\langle\varphi_\nu^i|\varphi_\nu^i\rangle
\end{align}
根据引理1,由于$\nu$是正数或零,于是右矢$a^\dagger|\varphi_\nu^i\rangle$的模永远不为零,故该右矢也永远不为零。\par 
(2)类似引理2的证明,要证明$a^\dagger|\varphi_\nu^i\rangle$是$N$的本征矢,只需利用算符关系式\eqref{b17b}:
\begin{align}
	[N,a^\dagger]|\varphi_\nu^i\rangle&=a^\dagger|\varphi_\nu^i\rangle\nonumber\\
	Na^\dagger|\varphi_\nu^i\rangle=a^\dagger N|\varphi_\nu^i\rangle+a^\dagger&|\varphi_\nu^i\rangle=(\nu+1)a^\dagger|\varphi_\nu^i\rangle
\end{align}
\subsubsection{$N$的谱由非负整数构成}
我们考虑$N$的任意一个本征值$\nu$,以及与他对应的非零本征矢$|\varphi_\nu^i\rangle$。\par 
假设$\nu$不是整数,我们总能找到这样的整数$n\geqslant0$,使得
\begin{equation}
	n<\nu<n+1
	\label{b30b30}
\end{equation}
我们再来考虑下面的矢量序列
\begin{equation}
	|\varphi_\nu^i\rangle,a|\varphi_\nu^i\rangle,\cdots,a^n|\varphi_\nu^i\rangle
\end{equation}
根据引理2,这个序列中的每一个矢量$a^p|\varphi_\nu^i\rangle(0\leqslant p\leqslant0)$都不为零,而且是$N$的本征矢,属于本征值$\nu-p$,假设根据$|\varphi_\nu^i\rangle$不为零,$a|\varphi_\nu^i\rangle$也不为零,他对应于$N$的本征值$\nu-1,\cdots;a^{p-1}|\varphi_\nu^i\rangle$是$N$的本征矢,属于本征值$\nu-p+1$,这是一个严格的正数,因为$p\leqslant n$而且$\nu >n$,将$a$作用于$a^{p-1}|\varphi_\nu^i\rangle$便可得到$a^p|\varphi_\nu^i\rangle$。\par 
现在将算符$a$作用于右矢$a^n|\varphi_\nu^i\rangle$。根据\eqref{b30b30},$\nu-n>0$,故将$a$作用于$a^n|\varphi_\nu^i\rangle$;此外,根据引理2,$a^{n+1}|\varphi_\nu^i\rangle$也是$N$的本征矢,属于本征值$\nu-n-1$,根据\eqref{b30b30},这是一个严格的负数。因而,如果$\nu$不是整数,我们便可以构成$N$的一个非零本征矢,他所对应的本征值是严格的负数。根据引理1,这是不可能的,因此,我们必须放弃$\nu$不是整数的假设。\par
如果$n$是正整数或零,而
\begin{equation}
	\nu=n
\end{equation}
在上述矢量序列中,$a^n|\varphi_\nu^i\rangle$不为零,而且是$N$的本征矢,属于本征值零。于是,根据引理2,我们有
\begin{equation}
	a^{n+1}|\varphi_\nu^i\rangle=0
\end{equation}
由此可见,若$n$是整数,则将算符$a$迭次作用于$|\varphi_\nu^i\rangle$所得到的矢量序列是有限的;因此,我们永远不可能得到$N$的属于负本征值的非零本征矢。\par 
归结起来,$\nu$只能是非负整数$n$。\par 
现在我们便可以用引理3证明:$N$的谱实际上包含全体正整数和零。在上面我们已经构成了$N$的一个本征矢($a^n|\varphi_\nu^i\rangle$),他属于本征值零;只需将算符$(a^\dagger)^k$作用于这个矢量,便可以得到$N$的属于本征值$k$的本征矢,这里的$k$是一个任意的正整数。\par 
如果参考\eqref{b19b19},我们便可以肯定$H$的本征值应为:
\begin{equation}
	E_n=\left( n+\dfrac{1}{2}\right)\hbar\omega 
\end{equation}
其中$n=0,1,2,\cdots$,由此可见,在量子力学中,谐振子的能量是量子化的,我们得到了与多项式解法相同的结论。\par 
我们还要对算符$a$和$a^\dagger$进行解释。将算符$a$作用于$H$的属于本征值$E_n=(n+1/2)\hbar\omega$的本征矢$|\varphi_\nu^i\rangle$,我们便可以得到属于本征值$E_{n-1}=(n+1/2)\hbar\omega-\hbar\omega$的一个本征矢,$a^\dagger$的作用同样方式给出能量$E_{n+1}=(n+1/2)\hbar\omega+\hbar\omega$。\par 
所以我们称$a$为湮没算符,$a^\dagger$为产生算符,他们对$N$的本征矢的作用是使得一个能量子$\hbar\omega$消失或产生。
\subsubsection{本征值的简并度}
我们将证明,一维谐振子的能级是非简并的。根据引理2,$N$的与本征值$n=0$相联系的所有本征态,都应满足下列方程
\begin{equation}
	a|\varphi^i_0\rangle=0
	\label{b35b35}
\end{equation}
因而,为了考虑能级$E_0$的简并度,只需考虑满足\eqref{b35b35}的线性无关的右矢有多少个即可。\par 
利用$a$的定义以及关系式\eqref{b1b1},可将\eqref{b35b35}写成下列形式:
\begin{equation}
	\dfrac{1}{\sqrt{2}}\left[ \sqrt{\dfrac{m\omega}{\hbar}}X+\dfrac{\mathrm{i}}{\sqrt{m\hbar\omega}}P\right] |\varphi^i_0\rangle=0
\end{equation}
在${|x\rangle}$表象下,此式变成
\begin{equation}
	\left( \dfrac{m\omega}{\hbar}x+\dfrac{\mathrm{d}}{\mathrm{d}x}\right) \varphi^i_0(x)=0
	\label{b37b37}
\end{equation}
式中
\begin{equation}
	\varphi^i_0(x)=\langle x|\varphi^i_0\rangle
\end{equation}
这个一阶微分方程的通解为
\begin{equation}
	\varphi^i_0(x)=ce^{-\frac{1}{2}\frac{m\omega}{\hbar}x^2}
\end{equation}
因此,方程\eqref{b37b37}的所有解彼此成比例,这就是说,除一个倍乘因子以外,满足\eqref{b35b35}的右矢$|\varphi_0\rangle$只有一个;可见基态能级$E_0=\hbar\omega/2$是非简并的。\par 
接下来我们用递推的方法证明所有其他能级也都是非简并的。\par 
为此,只需再证明:如果能级$E_n=(n+1/2)\hbar\omega$是非简并的,则能级$E_{n+1}=(n+1+1/2)\hbar\omega$也是非简并的。因此,我们假设,除一个倍乘因子以外,只有一个右矢$|\varphi_n\rangle$可以满足关系:
\begin{equation}
	N|\varphi_n\rangle=n|\varphi_n\rangle
\end{equation}
再考虑对应于本征值$n+1$的一个本征矢$|\varphi^i_{n+1}\rangle$
\begin{equation}
	N|\varphi^i_{n+1}\rangle=(n+1)|\varphi^i_{n+1}\rangle
	\label{b41b41}
\end{equation}
我们知道,右矢$a|\varphi^i_{n+1}\rangle$不为零,而且是$N$的本征矢,属于本征值$n$。根据假设,这个本征值是非简并的,因而存在一个数$c^i$,使得
\begin{equation}
	a|\varphi^i_{n+1}\rangle=c^i|\varphi^i_{n}\rangle
\end{equation}
将算符$a^\dagger$作用于此式两端可将$|\varphi^i_{n+1}\rangle$解出来
\begin{equation}
	a^\dagger a|\varphi^i_{n+1}\rangle=c^ia^\dagger|\varphi^i_{n}\rangle
\end{equation}
考虑到\eqref{b13b13}\eqref{b41b41},也就是
\begin{equation}
	|\varphi^i_{n+1}\rangle=\dfrac{c^i}{n+1}a^\dagger|\varphi_n\rangle
\end{equation}
我们已经知道$a^\dagger|\varphi_n\rangle$是$N$的本征矢属于本征值$(n+1)$;我们看到,属于本征值$(n+1)$的所有的右矢$|\varphi^i_{n+1}\rangle$都与$a^\dagger|\varphi_n\rangle$成比例;因而他们也互相成比例,这就是说,本征值$(n+1)$是非简并的。\par 
\subsection{哈密顿算符的本征态}
我们承认$N$和$H$都是观察算符,也就是说,他们各自的本征矢的集合都构成一维问题中的一个粒子的态空间$\mathscr{E}_x$的一个基。由于$N$的每一个本征值都是非简并的,故$N$(或$H$)本身就构成$\mathscr{E}_x$空间的一个CSCO。
\subsubsection{基矢量表为$|\varphi_0\rangle$的函数}
与$n=0$相联系的矢量$|\varphi_0\rangle$是$\mathscr{E}_x$空间的矢量,他满足关系
\begin{equation}
	a|\varphi_0\rangle=0
\end{equation}
除倍乘因子以外,这个矢量是确定的,我们假设$|\varphi_0\rangle$已归一化,则他的不确定性只限于一个形如$e^{\mathrm{i}\theta}$的全局相位因子。\par 
根据引理3,对应于$n=1$的矢量$|\varphi_1\rangle$与矢量$a^\dagger|\varphi_0\rangle$成比例:
\begin{equation}
	|\varphi_1\rangle=c_1a^\dagger|\varphi_0\rangle
	\label{c2c2}
\end{equation}
为了确定$c_1$,我们规定$|\varphi_1\rangle$已归一化,并选择$|\varphi_1\rangle$的相位使得$c_1$为正实数。根据\eqref{c2c2},$|\varphi_1\rangle$的模平方为:
\begin{align}
	\langle\varphi_1|\varphi_1\rangle&=|c_1|^2\langle\varphi_0|aa^\dagger|\varphi_0\rangle\nonumber\\
									 &=|c_1|^2\langle\varphi_0|(a^\dagger a+1)|\varphi_0\rangle
\end{align}
由于$|\varphi_0\rangle$是$N=a^\dagger a$的属于本征值零的已归一的本征矢,故得
\begin{equation}
	\langle\varphi_1|\varphi_1\rangle=|c_1|^2=1
\end{equation}
按照上面相位的规定,应取$c_1=1$,因而
\begin{equation}
	|\varphi_1\rangle=a^\dagger|\varphi_0\rangle
\end{equation}

同样的,我们可以从$|\varphi_1\rangle$出发构建$|\varphi_2\rangle$
\begin{equation}
	|\varphi_2\rangle=c_2a^\dagger|\varphi_1\rangle
\end{equation}
于是
\begin{equation}
	\langle\varphi_2|\varphi_2\rangle=2|c_2|^2=1
\end{equation}
便可得
\begin{equation}
	|\psi_2\rangle=\dfrac{1}{\sqrt{2}}a^\dagger|\varphi_1\rangle=\dfrac{1}{\sqrt{2}}(a^\dagger)^2|\varphi_0\rangle
\end{equation}

这种做法很容易推广。如果已知归一化的$|\varphi_{n-1}\rangle$,那么,已归一化的矢量$|\varphi_n\rangle$就可以写作
\begin{equation}
	|\varphi_n\rangle=c_na^\dagger|\varphi_{n-1}\rangle
\end{equation}
由于
\begin{align}
	\langle\varphi_n|\varphi_n\rangle&=|c_n|^2\langle\varphi_{n-1}|aa^\dagger|\varphi_{n-1}\rangle\nonumber\\
	&=n|c_n|^2=1
\end{align}
按上面我们相位的规定,应该取
\begin{equation}
	c_n=\dfrac{1}{\sqrt{n}}
\end{equation}
相继选择相位,我们可以从$|\varphi_0\rangle$出发得到所有的$|\varphi_n\rangle$
\begin{align}
	|\varphi_n\rangle&=\dfrac{1}{\sqrt{n}}a^\dagger|\varphi_{n-1}\rangle=\dfrac{1}{\sqrt{n}}\dfrac{1}{\sqrt{n-1}}(a^\dagger)^2|\varphi_{n-2}\rangle=\cdots\nonumber\\
	&=\dfrac{1}{\sqrt{n}}\dfrac{1}{\sqrt{n-1}}\cdots\dfrac{1}{\sqrt{2}}(a^\dagger)^n|\varphi_{0}\rangle
\end{align}
或写作
\begin{equation}
	|\varphi_n\rangle=\dfrac{1}{\sqrt{n!}}(a^\dagger)^n|\varphi_{0}\rangle
	\label{c13c13}
\end{equation}
\subsubsection{正交归一关系式和封闭性关系式}
$H$既然是厄米算符,那么,与不同的$n$值对应的那些右矢$|\varphi_n\rangle$必是互相正交的;此外,由于每个右矢都已归一化,因此他们满足正交归一关系式:
\begin{equation}
	\langle\varphi_{n^\prime}|\varphi_n\rangle=\delta_{nn^\prime}
\end{equation}

另一方面,$H$又是一个观察算符,所以全体$|\varphi_n\rangle$的集合构成$\mathscr{E}_x$空间中的一个基,这一点由封闭性关系
\begin{equation}
	\sum\limits_{n}|\varphi_n\rangle\langle\varphi_n|=1
\end{equation}
来表示。
\subsubsection{各算符的作用}
观察算符$X$和$P$都是算符$a$和$a^\dagger$的线性组合。因而所有的物理量都可以表示为$a$和$a^\dagger$的函数。他们在表象${|\varphi_n\rangle}$中各基矢量的作用又下列公式表示
\begin{subequations}
	\begin{equation}
		a^\dagger|\varphi_n\rangle=\sqrt{n+1}|\varphi_{n+1}\rangle
	\end{equation}
	\begin{equation}
		a|\varphi_n\rangle=\sqrt{n}|\varphi_{n-1}\rangle
	\end{equation}
\end{subequations}

从上式出发,利用\eqref{b1b1}\eqref{b7b7},我们可以得$X|\varphi_n\rangle$与$P|\varphi_n\rangle$的表示式
\begin{subequations}
	\begin{align}
		X|\varphi_n\rangle&=\sqrt{\dfrac{\hbar}{m\omega}}\dfrac{1}{\sqrt{2}}(a^\dagger+a)|\varphi_n\rangle\nonumber\\
		&=\sqrt{\dfrac{\hbar}{2m\omega}}[\sqrt{n+1}|\varphi_{n+1}\rangle+\sqrt{n}|\varphi_{n-1}\rangle]
	\end{align}
	\begin{align}
		P|\varphi_n\rangle&=\sqrt{m\hbar\omega}\dfrac{\mathrm{i}}{\sqrt{2}}(a^\dagger-a)|\varphi_n\rangle\nonumber\\
		&=\mathrm{i}\sqrt{\dfrac{m\hbar\omega}{2}}[\sqrt{n+1}|\varphi_{n+1}\rangle+\sqrt{n}|\varphi_{n-1}\rangle]
	\end{align}
\end{subequations}
从而,算符$a,a^\dagger,X,P$在表象${|\varphi_n\rangle}$中的矩阵元分别为
\begin{subequations}
	\begin{equation}
		\langle\varphi_{n^\prime}|a|\varphi_n\rangle=\sqrt{n}\delta_{n^\prime,n-1}
	\end{equation}
	\begin{equation}
		\langle\varphi_{n^\prime}|a^\dagger|\varphi_n\rangle=\sqrt{n+1}\delta_{n^\prime,n+1}
	\end{equation}
	\begin{equation}
		\langle\varphi_{n^\prime}|X|\varphi_n\rangle=\sqrt{\dfrac{\hbar}{2m\omega}}[\sqrt{n+1}\delta_{n^\prime,n+1}+\sqrt{n}\delta_{n^\prime,n-1}]
	\end{equation}
	\begin{equation}
		\langle\varphi_{n^\prime}|P|\varphi_n\rangle=\mathrm{i}\sqrt{\dfrac{m\hbar\omega}{2}}[\sqrt{n+1}\delta_{n^\prime,n+1}-\sqrt{n}\delta_{n^\prime,n-1}]
	\end{equation}
\end{subequations}
\subsection{与定态相联系的波函数}
现在采用${|x\rangle}$表象。现在已经求出函数$\varphi_0(x)$,他表示基态$|\varphi_0\rangle$
\begin{equation}
	\varphi_0(x)=\langle x|\varphi_0\rangle=\left( \dfrac{m\omega}{\pi\hbar}\right) ^{1/4}e^{-\frac{1}{2}\frac{m\omega}{\hbar}x^2}
\end{equation}
指数函数前的常数保证归一化。\par 
为了求得与谐振子的其他定态相联系的波函数,只需利用\eqref{c13c13}与以下事实:在表象${|x\rangle}$中,$X$表示用$x$去倍乘,而$P$相当于$\dfrac{\hbar}{\mathrm{i}}\dfrac{\mathrm{d}}{\mathrm{d}x}$,于是
\begin{align}
	\varphi_n(x)&=\langle x|\varphi_n\rangle=\dfrac{1}{\sqrt{n!}}=\langle x|(a^\dagger)^n|\varphi_n\rangle\nonumber\\
	&=\dfrac{1}{\sqrt{n!}}\dfrac{1}{\sqrt{2}}\left[ \sqrt{\dfrac{m\omega}{\hbar}}x-\sqrt{\dfrac{\hbar}{m\omega}}\dfrac{\mathrm{d}}{\mathrm{d}x}\right] ^n\varphi_0(x)
\end{align}
或写作
\begin{equation}
	\varphi_n(x)=\left[ \dfrac{1}{2^nn!}\left( \dfrac{\hbar}{m\omega}\right) ^n\right] ^{1/2}\left( \dfrac{m\omega}{\pi\hbar}\right) ^{1/4}\left[ \dfrac{m\omega}{\hbar}x-\dfrac{\mathrm{d}}{\mathrm{d}x}\right] ^ne^{-\frac{1}{2}\frac{m\omega}{\hbar}x^2}
\end{equation}
$\varphi_n(x)$就是函数$e^{-\frac{1}{2}\frac{m\omega}{\hbar}x^2}$与一个次数为$n$,宇称为$(-1)^n$的多项式乘积,这个多项式就是厄米多项式。
\section{各向同性的三维谐振子}
\subsection{哈密顿算符}
考虑一个无自旋粒子,质量为$m$,可在三维空间中运动,受向心力的作用,力的大小正比于粒子至$O$点的距离
\begin{equation}
	F=-kr
\end{equation}

这个力场下的势能
\begin{equation}
	V(r)=\dfrac{1}{2}kr^2=\dfrac{1}{2}m\omega^2r^2
\end{equation}
式中角频率为
\begin{equation}
	\omega=\sqrt{\dfrac{k}{m}}
\end{equation}
于是,经典哈密顿函数为
\begin{equation}
	\mathscr{H}(r,p)=\dfrac{p^2}{2m}+\dfrac{1}{2}m\omega^2r^2
\end{equation}
根据量子化规则,哈密顿算符为i
\begin{equation}
	H=\dfrac{P^2}{2m}+\dfrac{1}{2}m\omega^2R^2
\end{equation}
由于哈密顿算符与时间无关,我们将解他的本征值方程
\begin{equation}
	H|\varphi\rangle=E|\varphi\rangle
	\label{V6}
\end{equation}
这里的$|\varphi\rangle$属于三维空间中运动的粒子的态空间$\mathscr{E}_r$。\par 
由于V(r)实际上只依赖粒子到原点的距离,所以这个谐振子是各向同性的,势能可以写作
\begin{equation}
	V(r)=\dfrac{m}{2}(\omega^2_xx^2+\omega^2_yy^2+\omega^2_zz^2)
\end{equation}
\subsection{直角坐标系下变量的分离}
我们将态空间$\mathscr{E}_r$看做一个张量积
\begin{equation}
	\mathscr{E}_r=\mathscr{E}_x\otimes\mathscr{E}_y\otimes\mathscr{E}_z
\end{equation}

哈密顿算符还可以写作
\begin{align}
	H&=\dfrac{1}{2m}(P_x^2+P_y^2+P_z^2)+\dfrac{1}{2}m\omega^2(X^2+Y^2+Z^2)\nonumber\\
	 &=H_x+H_y+H_z
	 \label{V9}
\end{align}

$H_x,H_y,H_z$互相对易,所以,其中的每一个都与他们的总和$H$对易。因此,只要解出本征方程\eqref{V6},我们可寻求$H$的诸本征矢,他们也是$H_x,H_y,H_z$的本征矢。我们已经知道$\mathscr{E}_x$中$H_x$的本征矢和本征值,以及其他两个空间
\begin{subequations}
	\begin{equation}
		H_x|\varphi_{n_x}\rangle=\left( n_x+\dfrac{1}{2}\right) \hbar\omega|\varphi_{n_x}\rangle\quad|\varphi_{n_x}\rangle\in\mathscr{E}_x
	\end{equation}
	\begin{equation}
		H_y|\varphi_{n_y}\rangle=\left( n_y+\dfrac{1}{2}\right) \hbar\omega|\varphi_{n_y}\rangle\quad|\varphi_{n_y}\rangle\in\mathscr{E}_y
	\end{equation}
	\begin{equation}
		H_z|\varphi_{n_z}\rangle=\left( n_z+\dfrac{1}{2}\right) \hbar\omega|\varphi_{n_z}\rangle\quad|\varphi_{n_z}\rangle\in\mathscr{E}_z
	\end{equation}
	\label{V11}
\end{subequations}
共同的本征矢为
\begin{equation}
	|\varphi_{n_x,n_y,n_z}\rangle=|\varphi_{n_x}\rangle|\varphi_{n_y}\rangle|\varphi_{n_z}\rangle
	\label{V12}
\end{equation}
根据\eqref{V9}\eqref{V11}
\begin{equation}
	H|\varphi_{n_x,n_y,n_z}\rangle=\left( n_x+n_y+n_z+\dfrac{3}{2}\right) \hbar\omega|\varphi_{n_x,n_y,n_z}\rangle
	\label{V13}
\end{equation}
这就是说,$H$的本征矢就是$H_x,H_y,H_z$各种的本征矢的张量积,而$H$的本征值就是这三个算符的本征值之和。\par
根据\eqref{V13},各向同性的三维谐振子的能级具有下列形式
\begin{equation}
	E_n=\left( n+\dfrac{3}{2}\right) \hbar\omega
\end{equation}
其中,$n$是0或正整数。\par 
现在引用三组产生、湮没算符
\begin{subequations}
	\begin{equation}
		a_x=\sqrt{\dfrac{m\omega}{2\hbar}}X+\dfrac{\mathrm{i}}{\sqrt{2m\hbar\omega}}P_x\quad a^\dagger_x=\sqrt{\dfrac{m\omega}{2\hbar}}X-\dfrac{\mathrm{i}}{\sqrt{2m\hbar\omega}}P_x
	\end{equation}
	\begin{equation}
		a_y=\sqrt{\dfrac{m\omega}{2\hbar}}Y+\dfrac{\mathrm{i}}{\sqrt{2m\hbar\omega}}P_y\quad a^\dagger_y=\sqrt{\dfrac{m\omega}{2\hbar}}Y-\dfrac{\mathrm{i}}{\sqrt{2m\hbar\omega}}P_y
	\end{equation}
	\begin{equation}
		a_z=\sqrt{\dfrac{m\omega}{2\hbar}}Z+\dfrac{\mathrm{i}}{\sqrt{2m\hbar\omega}}P_z\quad a^\dagger_z=\sqrt{\dfrac{m\omega}{2\hbar}}Z-\dfrac{\mathrm{i}}{\sqrt{2m\hbar\omega}}P_z
	\end{equation}
\end{subequations}
六个算符之间的非零对易子只是
\begin{equation}
	[a_x,a_x^\dagger]=[a_y,a_y^\dagger]=[a_z,a_z^\dagger]=1
\end{equation}
算符$a_x,a^\dagger_x$对态矢量$|\varphi_{n_x,n_y,n_z}\rangle$的作用为
\begin{subequations}
	\begin{align}
		a_x|\varphi_{n_x,n_y,n_z}\rangle&=(a_x|\varphi_{n_x}\rangle)|\varphi_{n_y}\rangle|\varphi_{n_z}\rangle\nonumber\\
		&=\sqrt{n_x}|\varphi_{n_x-1}\rangle|\varphi_{n_y}\rangle|\varphi_{n_z}\rangle\nonumber\\
		&=\sqrt{n_x}|\varphi_{n_x-1,n_y,n_z}\rangle
	\end{align}
	\begin{align}
		a_x^\dagger|\varphi_{n_x,n_y,n_z}\rangle&=(a_x^\dagger|\varphi_{n_x}\rangle)|\varphi_{n_y}\rangle|\varphi_{n_z}\rangle\nonumber\\
		&=\sqrt{n_x+1}|\varphi_{n_x+1}\rangle|\varphi_{n_y}\rangle|\varphi_{n_z}\rangle\nonumber\\
		&=\sqrt{n_x+1}|\varphi_{n_x+1,n_y,n_z}\rangle
	\end{align}
\end{subequations}

我们又知道
\begin{equation}
	|\varphi_{n_x}\rangle=\dfrac{1}{\sqrt{n_x!}}(a^\dagger_x)^{n_x}|\varphi_0\rangle
\end{equation}
其中$|\varphi_0\rangle$是$\mathscr{E}_x$中的右矢,他满足条件
\begin{equation}
	a_x|\varphi_0\rangle=0
\end{equation}
$|\varphi_{n_y}\rangle$和$|\varphi_{n_z}\rangle$分别在空间$\mathscr{E}_y$和$\mathscr{E}_z$中也有类型的关系式,因此根据\eqref{V12},有
\begin{equation}
	|\varphi_{n_x,n_y,n_z}\rangle=\dfrac{1}{\sqrt{n_x!n_y!n_z!}}(a^\dagger_x)^{n_x}(a^\dagger_y)^{n_y}(a^\dagger_z)^{n_z}|\varphi_{0,0,0}\rangle
\end{equation}
式中$|\varphi_{0,0,0}\rangle$是三个一维谐振子的基态的张量积,因而,他满足
\begin{equation}
	a_x|\varphi_{0,0,0}\rangle=a_y|\varphi_{0,0,0}\rangle=a_z|\varphi_{0,0,0}\rangle
\end{equation}
最后
\begin{equation}
	\langle r|\varphi_{0,0,0}\rangle=\left( \dfrac{m\omega}{\pi\hbar}\right) ^{3/4}e^{-\frac{m\omega}{2\hbar}(x^2+y^2+z^2)}
\end{equation}
\subsection{能级的简并度}
$H$不能单独构成一个CSCO,因为能级$E_n$是简并的。实际上,假设我们选定了$H$的一个本征值$E_n=(n+3/2)\hbar\omega$;这等于说已将$n$取定为某一个非负整数值。在基${|\varphi_{n_x,n_y,n_z}\rangle}$中,凡是满足条件
\begin{equation}
	n_x+n_y+n_z=n
\end{equation}
的所有右矢都是$H$的属于本征值$E_n$的本征矢。\par 
如果我们选定$n_x$,我们便有
\begin{equation}
	n_y+n_z=n-n_x
\end{equation}
于是,数值${n_y,n_z}$有$(n-n_x+1)$个可能性
\begin{equation}
	{n_y,n_z}={0,n-n_x},{1,n-n_x-1},\cdots,{n-n_x,0}
\end{equation}
因此,简并度为
\begin{equation}
	g_n=\sum\limits_{n_x=0}^{n}(n-n_x+1)
\end{equation}
不难算出
\begin{equation}
	g_n=(n+1)\sum\limits_{n_x=0}^{n}1-\sum\limits_{n_x=0}^{n}n_x=\dfrac{(n+1)(n+2)}{2}
\end{equation}
\section{二维谐振子}
\subsection{量子化}
\subsubsection{经典力学的处理}
如果粒子的势能只依赖于$x$和$y$,这种情况就退化为二维问题。我们假设这个势能可以写作
\begin{equation}
	V(x,y)=\dfrac{\mu}{2}\omega^2(x^2+y^2)
\end{equation}
式中,$\mu$是粒子的质量,$\omega$是一常数,于是,体系的哈密顿函数为
\begin{equation}
	\mathscr{H}=\mathscr{H}_{xy}+\mathscr{H}_z
\end{equation}
其中
\begin{align}
	\mathscr{H}_{xy}&=\dfrac{1}{2\mu}(p_x^2+p_y^2)+\dfrac{1}{2}\mu\omega^2(x^2+y^2)\nonumber\\
	\mathscr{H}_z&=\dfrac{1}{2\mu}p_z^2
\end{align}
\subsubsection{量子化处理}
于是
\begin{equation}
	H|\varphi\rangle=(H_{xy}+H_z)|\varphi\rangle
\end{equation}
其中
\begin{subequations}
	\begin{equation}
		H_{xy}=\dfrac{P_x^2+P_y^2}{2\mu}+\dfrac{1}{2}\mu\omega^2(X^2+Y^2)
	\end{equation}
	\begin{equation}
		H_z=\dfrac{P_z^2}{2\mu}
	\end{equation}
\end{subequations}

我们知道,可以选择一个由$H$的形如
\begin{equation}
	|\varphi\rangle=|\varphi_{xy}\rangle\otimes|\varphi_z\rangle
\end{equation}
的本征矢所构成的基:
\begin{equation}
	H_{xy}|\varphi_{xy}\rangle=E_{xy}|\varphi_{xy}\rangle
	\label{L14}
\end{equation}
\begin{equation}
	H_z|\varphi_z\rangle=E_z|\varphi_z\rangle
	\label{L15}
\end{equation}
因此,总能量为
\begin{equation}
	E=E_{xy}+E_z
\end{equation}
但是,\eqref{L15}所确定的态就是一维问题中一个自由粒子的定态,此方程可以解出
\begin{equation}
	\langle z|\varphi_z\rangle=\dfrac{1}{\sqrt{2\pi\hbar}}e^{\mathrm{i}p_zz/\hbar}
\end{equation}
而且
\begin{equation}
	E_z=\dfrac{p_z^2}{2\mu}
\end{equation}
于是问题归结为求方程\eqref{L14}的解,即求一个二维谐振子的诸定态及对应的能量。
\subsection{将定态按量子数$n_x$与$n_y$分类}
可以将算符$H_{xy}$写作
\begin{equation}
	H_{xy}=H_x+H_y
\end{equation}
这里$H_x,H_y$为
\begin{align}
	H_x&=\dfrac{P_x^2}{2\mu}+\dfrac{1}{2}\mu\omega^2X^2\nonumber\\
	H_y&=\dfrac{P_y^2}{2\mu}+\dfrac{1}{2}\mu\omega^2Y^2
\end{align}
因此,可将$H_{xy}$的本征态取
\begin{equation}
	|\varphi_{n_x,n_y}\rangle=|\varphi_{n_x}\rangle\otimes|\varphi_{n_y}\rangle
\end{equation}
对应能量为
\begin{align}
	E_{xy}&=\left( n_x+\dfrac{1}{2}\right) \hbar\omega+\left( n_y+\dfrac{1}{2}\right) \hbar\omega\nonumber\\
	&=(n_x+n_y+1)\hbar\omega
	\label{L22}
\end{align}
根据一维谐振子的性质,在空间$\mathscr{E}_x$在$E_x$是非简并的;在空间$\mathscr{E}_y$在$E_y$是非简并的。因此,对于每一对数${n_x,n_y}$,在空间$\mathscr{E}_{xy}$中都有一个唯一的矢量$|\varphi_{n_x,n_y}\rangle$与之对应,这就是说,$H_x$和$H_y$构成$\mathscr{E}_{xy}$中的CSCO。\par 
$a_x$和$a_y$定义为
\begin{align}
	a_x&=\dfrac{1}{\sqrt{2}}\left( \beta X+\mathrm{i}\dfrac{P_x}{\beta\hbar}\right) \nonumber\\
	a_y&=\dfrac{1}{\sqrt{2}}\left( \beta Y+\mathrm{i}\dfrac{P_y}{\beta\hbar}\right) 
\end{align}
两式中
\begin{equation}
	\beta=\sqrt{\dfrac{\mu\omega}{\hbar}}
\end{equation}
因为$a_x$与$a_y$在不同的空间$\mathscr{E}_x$与$\mathscr{E}_y$中起作用,所以$a_x,a_y,a_x^\dagger,a_y^\dagger$这四个算符之间的不为零的对易子只有
\begin{equation}
	[a_x,a_x^\dagger]=[a_y,a_y^\dagger]=1
	\label{L25}
\end{equation}
此外,算符$N_x$和$N_y$为
\begin{align}
	N_x&=a_x^\dagger a_x\nonumber\\
	N_y&=a_y^\dagger a_y
\end{align}
于是$H_{xy}$可以写成
\begin{equation}
	H_{xy}=H_x+H_y=(N_x+N_y+1)\hbar\omega
\end{equation}
显然有
\begin{align}
	N_x|\varphi_{n_x,n_y}\rangle&=n_x|\varphi_{n_x,n_y}\rangle\nonumber\\
	N_y|\varphi_{n_x,n_y}\rangle&=n_y|\varphi_{n_x,n_y}\rangle
\end{align}

基态$|\varphi_{0,0}\rangle$由下式给出
\begin{equation}
	|\varphi_{0,0}\rangle=|\varphi_{n_x=0}\rangle\otimes|\varphi_{n_y=0}\rangle
\end{equation}
将算符$a_x^\dagger$和$a_y^\dagger$相继作用于$|\varphi_{0,0}\rangle$,便可得态$|\varphi_{n_x,n_y}\rangle$
\begin{equation}
	|\varphi_{n_x,n_y}\rangle=\dfrac{1}{\sqrt{n_x!n_y!}}(a_x^\dagger)^{n_x}(a_y^\dagger)^{n_y}|\varphi_{0,0}\rangle
\end{equation}
用$\varphi_{n_y}(y)$乘$\varphi_{n_x}(x)$,所得之积就是对应的波函数
\begin{equation}
	\varphi_{n_x,n_y}(x,y)=\dfrac{\beta}{\sqrt{\pi(2)^{n_x+n_y}(n_x)!(n_y)!}}e^{-\beta^2(x^2+y^2)/2}H_{n_x}(\beta x)H_{n_y}(\beta y)
\end{equation}

由\eqref{L22}可以看出,$H_{xy}$的本征值具有如下形式
\begin{equation}
	E_{xy}=E_n=(n+1)\hbar\omega
\end{equation}
可见在空间$\mathscr{E}_{xy}$中本征值$E_n$是$(n+1)$重简并的,因此$H_{xy}$本身不能单独构成一个CSCO。而${H_{xy},H_x},{H_{xy},H_y},{H_x,H_y}$都是CSCO。
\subsection{将定态按角动量分类}
\subsubsection{算符$L_z$的意义及性质}
在这个问题中,$Ox$轴与$Oy$轴具有优越性,因为势能在围绕$Oz$轴的旋转中是不变的。为了更好的利用这个对称性,我们现在考虑
\begin{equation}
	L_z=XP_y-YP_x
\end{equation}
用$a_x,a_x^\dagger$表示$X$和$P_x$,同样$P_y$也可以类似表示
\begin{equation}
	L_z=\mathrm{i}\hbar(a_xa_y^\dagger-a_x^\dagger a_y)
	\label{L36}
\end{equation}
利用这些算符,可将$H_{xy}$表示为
\begin{equation}
	H_{xy}=(a_x^\dagger a_x+a_y^\dagger a_y+1)\hbar\omega
\end{equation}
由于
\begin{align}
	[a_xa_y^\dagger,a_x^\dagger a_x+a_y^\dagger a_y]&=a_xa_y^\dagger-a_xa_y^\dagger=0\nonumber\\
	[a_x^\dagger a_y,a_x^\dagger a_x+a_y^\dagger a_y]&=-a_x^\dagger a_y+a_x^\dagger a_y=0
\end{align}
从而可以得出
\begin{equation}
	[H_{xy},L_z]=0
\end{equation}
因此,我们要去寻找由$H_{xy}$和$L_z$的共同本征矢所构成的基。
\subsubsection{右旋圆量子和左旋圆量子}
我们引入两个算符$a_d,a_g$
\begin{align}
	a_d&=\dfrac{1}{\sqrt{2}}(a_x-\mathrm{i}a_y)\nonumber\\
	a_g&=\dfrac{1}{\sqrt{2}}(a_x+\mathrm{i}a_y)
\end{align}
将算符$a_d$(或$a_g$)作用于矢量$|\varphi_{n_x,n_y}\rangle$所得到的态是$|\varphi_{n_x-1,n_y}\rangle$和$|\varphi_{n_x,n_y-1}\rangle$的一个线性组合,也就是少了一个能量子$\hbar\omega$的定态。实际上,我们将会看到,算符$a_d$(或$a_g$)非常类似于算符$a_x$(或$a_y$),而且我们可以将$a_d$与$a_g$分别解释为一个“右旋圆量子”与一个“左旋圆量子”的湮没算符。\par 
$a_d,a_g,a_d^\dagger,a_g^\dagger$四个算符之间的不为零的对易子只有
\begin{equation}
	[a_d,a_d^\dagger]=[a_g,a_g^\dagger]=1
\end{equation}
这个关系与\eqref{L25}非常相似。此外,由于
\begin{align}
	a_d^\dagger a_d&=\dfrac{1}{2}(a_x^\dagger a_x+a_y^\dagger a_y-\mathrm{i}a_x^\dagger a_y+\mathrm{i}a_xa_y^\dagger)\nonumber\\
	a_g^\dagger a_g&=\dfrac{1}{2}(a_x^\dagger a_x+a_y^\dagger a_y+\mathrm{i}a_x^\dagger a_y-\mathrm{i}a_xa_y^\dagger)
\end{align}
从而有
\begin{equation}
	H_{xy}=(a_d^\dagger a_d+a_g^\dagger a_g+1)\hbar\omega
	\label{L43}
\end{equation}
此外,考虑\eqref{L36},还可以看出
\begin{equation}
	L_z=\hbar(a_d^\dagger a_d-a_g^\dagger a_g)
	\label{L44}
\end{equation}
如果引入$N_g$与$N_d$
\begin{align}
	N_d&=a_d^\dagger a_d\nonumber\\
	N_g&=a_g^\dagger a_g
\end{align}
则\eqref{L43}与\eqref{L44}为
\begin{align}
	H_{xy}&=(N_d+N_g+1)\hbar\omega\nonumber\\
	L_z&=\hbar(N_d-N_g)
	\label{L4646}
\end{align}
\subsubsection{具有完全确定的角动量的定态}
现在,我们可以使用算符$a_d$和$a_g$来进行类似于上面使用$a_x$和$a_y$所做的推理。这样便可推知:算符$N_d$和$N_g$的谱由全体正整数或零构成;此外,给出了这样的一对整数${n_d.n_g}$便唯一的决定了算符$N_d$和$N_g$的属于这组本征值的共同本征矢
\begin{equation}
	|\chi_{n_d,n_g}\rangle=\dfrac{1}{\sqrt{(n_d)!(n_g)!}}(a_d^\dagger)^{n_d}(a_g^\dagger)^{n_g}|\varphi_{0,0}\rangle
\end{equation}
因此$N_d$和$N_g$在空间$\mathscr{E}_{xy}$中构成一个CSCO。利用\eqref{L46},可以看出,矢量$|\xi_{n_d,n_g}\rangle$也是算符$H_{xy}$和$L_z$的本征矢,属于本征值$(n+1)\hbar\omega$和$m\hbar$,这里$n$与$m$由下式给出
\begin{align}
	n&=n_d+n_g\nonumber\\
	m&=n_d-n_g
\end{align}
算符$a_d^\dagger$对矢量$|\xi_{n_d,n_g}\rangle$的作用所给出的态多了一个量子,由于$m$的值已经增大了1,我们便须给出这个态增添一个角动量$+\hbar$(这对应于绕$Oz$轴沿逆时针方向的旋转);同样的,算符$a_g^\dagger$所给出的态也多了一个量子,角动量的改变为$-\hbar$(这对应于顺时针旋转)。\par 
由于$n_d$和$n_g$都是任意正整数(或零),于是我们又得到了前一段的结果:$H_{xy}$的本征值的形式为$(n+1)\hbar\omega$,其中$n$为正整数或零;这些本征值的简并度为$(n+1)$,这是因为,$n$的值取定之后,可以有
\begin{align}
	n_d&=n\quad\quad n_g=0\nonumber\\
	n_d=&n-1\quad\quad n_g=1\nonumber\\
	&\vdots\quad\quad\quad\quad\quad\vdots\nonumber\\
	n_d&=0\quad\quad n_g=n
	\label{L49}
\end{align}

另一方面,我们看到算符$L_z$的本征值的形式为$m\hbar$,这里$m$为正的或负的整数或零。此外,从\eqref{L49}可以推知$m$的哪些值与$n$的一个给定值相联系。例如,对于基态,$n_d=0,n_g=1$,这就决定了$m=\pm 1$或$m=-1$,在一般情况下,对于一个给定的能级$(n+1)\hbar\omega$,$m$的可能值为
\begin{equation}
	m=n,n-2,n-4,\cdots,-n+2,-n
\end{equation}
由此便可推知,对于$n$和$m$的一对值,对应着一个唯一的矢量
\begin{equation}
	|\chi_{n_d=\frac{n+m}{2},n_g=\frac{n-m}{2}}\rangle
\end{equation}
因此,$H$和$L_z$在空间$\mathscr{E}_{xy}$中构成一个CSCO。
\end{document}